\paragraph{Pour conclure, la réalisation du projet Smashstein s’est révélée comme une expérience très instructive et enrichissante en tout point.}

\paragraph{Premièrement, elle nous aura permis de réaliser un véritable jeu vidéo dans des conditions proches de celles du travail en entreprise puisqu’au sein de la relativement grande équipe de ce projet, chacun a pu réellement se spécialiser afin de valoriser au mieux ses compétences propres et travailler de manière d’autant plus efficace.}

\paragraph{Par ailleurs ce projet a été l’occasion d’approfondir nos connaissances et de nous aguerrir dans divers domaines. Bien entendu, la programmation C++ a joué un rôle central et nous avons pu grandement pratiquer ce langage en découvrant au passage de nouvelles bibliothèques. L’utilisation des logiciels de modélisation \textit{3D} que sont \textit{Blender} et \textit{3DS MAX} a également été cruciale. Cependant, l’aspect le plus intéressant reste la découverte des mécanismes de combinaison de ces domaines qui n’avaient été abordés en cours que séparément jusqu’à présent.}

\paragraph{Enfin, s’il fallait émettre un avis sur notre travail, nous pourrions dire que, même si le résultat est encore perfectible sur certains points, \textit{Smashtein Garbage} remplit toutes les fonctionnalités demandées de manière satisfaisante et en apporte même de nouvelles. Nous sommes donc plutôt satisfaits du résultat, même si nous aurions aimé passer plus de temps sur la partie esthétique du jeu afin d’aboutir sur des décors et personnages plus jolis, mais aussi pourquoi pas de composer nos propres sons et mélodies à l’aide d’un logiciel comme \textit{Rosegarden}.}