\paragraph{Le jeu vidéo connait son premier âge d’or dans les années 80 grâce à des jeux comme Pac-Man ou Space Invaders et n’a depuis cessé de se développer à tous les niveaux, si bien qu’il est aujourd’hui en passe de devenir un média à part entière, au même titre que la télévision ou le cinéma.}

\paragraph{Il était donc important pour les étudiants en multimédia que nous sommes de pouvoir envisager le fonctionnement et la réalisation d’un jeu vidéo, ce que le projet Smashstein a été l’occasion de faire puisqu’il proposait de réaliser un jeu vidéo en 2.5D (3D où les déplacements du joueur sont cantonnés à un plan) doté d’une intelligence artificielle.}

\paragraph{Dans ce but nous avons procédé en deux phases principales, avec en premier lieu la \textbf{conception théorique}, comprenant la \textbf{répartition des tâches} au sein du groupe, l’\textbf{établissement d’un rétroplanning}, la réalisation de l’\textbf{architecture logicielle} et la \textbf{mise en place des conventions} de programmation. La seconde phase concerne la \textbf{réalisation} à proprement parler, à savoir la \textbf{modélisation des décors et personnages} et la \textbf{programmation en C++ de l'ensemble des fonctionnalités du jeu}, avec toutes les \textbf{difficultés techniques} qui l'accompagnent.}